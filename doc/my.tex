\documentclass[
    % draft,                             % 草稿模式
    aspectratio=169,                   % 使用 16:9 比例
]{beamer}
\mode<presentation>
\usetheme[
    % navigation=subsections,            % 使用子章节进度显示
    % lang=en,                           % 使用英文
    % cjk=true,                          % 使用CJK而不是ctex
    color=red,                         % 使用红色主题
    % pattern=all,                        % 使用全图案装饰
    % gbt=bibtex,                        % 使用 gbt (使用 bibtex 编译)
]{sjtubeamermin}
\usecolortheme[]{beaver}                 % 使用其他颜色主题
\addbibresource{ref.bib}               % gbt!=bibtex

\begin{document}
    % \institute[School of Electrical Engineering]{数学科学学院}   % 组织
    \institute[Department of Automation]{自动化系}   % 组织
    \logo{
        \includegraphics{cnlogored.pdf}  % 重定义 logo
    }
    \titlegraphic{                         % 标题图像
        \begin{stampbox}
            \includegraphics[width=0.3\textwidth]{head.png}
        \end{stampbox}
    }
    \title{Blockchain and Internet of Things 区块链和物联网}  % 标题
    \subtitle{Scalable architecture for IoT}         % 副标题
    \author{Pedro Hernández Rubio}                  % 作者
    \date{\today}                          % 日期  
    \maketitle                             % 创建标题页

\part{第一部分 Blockchain technologies applied to IoT architecture}

% 使用节目录
\AtBeginSection[]{
    \begin{frame}
        % \tableofcontents[currentsection]           % 传统节目录             
        \sectionpage                   % 节页
    \end{frame}
}

% 使用小节目录
% \AtBeginSubsection[]{                  % 在每小节开始
%     \begin{frame}
%         \tableofcontents[currentsection,currentsubsection]             % 传统小节目录             
%         \subsectionpage                % 小节页
%     \end{frame}
% }

\section{第 1 节 Blockchain applied to IoT}

\subsection{第 1 小节 Integrating Blockchain and IoT}

    \begin{frame}
        \frametitle{Integrating Blockchain and IoT}

        \paragraph{Research} \alert{Blockchain technologies} and \alert{P2P storage systems}

        \begin{itemize}
            \item Decentralized: IoT devices data are not required to be sent to centralized services \alert{(default trust)}: data can be stored securely on the peers in the network, anb blockchain assure its authenticity and control authorized accesses.
            \item Private-by-design IoT: providing privacy, robust systems and no single point of failure.
        \end{itemize}

    \end{frame}

\subsection{第 2 小节 Uses cases}

    \begin{frame}
        \frametitle{Use cases}

        \paragraph{列表} 这个\alert{幻灯片}有下面几项:

    \end{frame}

\subsection{第 3 小节 Blockchain features for IoT}

    \begin{frame}
        \frametitle{Blockchain features for IoT}

        \paragraph{列表} 这个\alert{幻灯片}有下面几项:

        \begin{itemize}
            \item Integrity
            \item Anonimity
            \item Adaptability
        \end{itemize}

    \end{frame}

    % \begin{frame}
    %     \frametitle{标题}
    %     \framesubtitle{子标题}

    %     \begin{equation}
    %         x^2+2x+1=(x+1)^2
    %     \end{equation}
        
    % \end{frame}

\section{第 2 节 IoT Access Control Management}

\subsection{第 1 小节 Scalability: elephant in the room}

    \begin{frame}
        \frametitle{Scalability: elephant in the room}
        
        \begin{block}{盒子}
            这是一个盒子\cite{conoscenti}
        \end{block}

        \begin{alertblock}{注意}
            注意内容
        \end{alertblock}

        \begin{exampleblock}{示例}
            示例内容
        \end{exampleblock}
    \end{frame}

\subsection{第 2 小节 IoT architecture (I)}

    \begin{frame}
        \frametitle{IoT architecture (I)}
        
        \begin{block}{盒子}
            这是一个盒子\cite{novo}
        \end{block}

        \begin{alertblock}{注意}
            注意内容
        \end{alertblock}

        \begin{exampleblock}{示例}
            示例内容
        \end{exampleblock}
    \end{frame}

\subsection{第 3 小节 IoT architecture (II)}

    \begin{frame}
        \frametitle{IoT architecture (II)}
        
        \begin{block}{盒子}
            这是一个盒子\cite{ouaddah}
        \end{block}

        \begin{alertblock}{注意}
            注意内容
        \end{alertblock}

        \begin{exampleblock}{示例}
            示例内容
        \end{exampleblock}
    \end{frame}

\subsection{第 4 小节 Limitations}

    \begin{frame}
        \frametitle{Limitations}
        
        \begin{block}{盒子}
            这是一个盒子\cite{novo}
        \end{block}

        \begin{alertblock}{注意}
            注意内容
        \end{alertblock}

        \begin{exampleblock}{示例}
            示例内容
        \end{exampleblock}
    \end{frame}

\subsection{第 5 小节 Security issues}

    \begin{frame}
        \frametitle{Security issues}
        
        \begin{block}{盒子}
            这是一个盒子\cite{novo}
        \end{block}

        \begin{alertblock}{注意}
            注意内容
        \end{alertblock}

        \begin{exampleblock}{示例}
            示例内容
        \end{exampleblock}
    \end{frame}

%     \begin{frame}[fragile]          % 注意添加 fragile 标记
%         \frametitle{代码块}
%         % 代码块参数:语言,标题
%         % 请减少代码初始的缩进
%         \begin{codeblock}[language=c++]{C++代码}
% #include<iostream>

% int main(){
%     // Console Output
%     std::cout << "Hello, SJTU!" << std::endl;
%     return 0;
% }
%         \end{codeblock}
%     \end{frame}

%     \begin{frame}
%         \frametitle{图}
%         \begin{figure}
%             \centering
%             \begin{stampbox}
%                 \includegraphics[height=0.3\textheight]{plant.jpg}
%             \end{stampbox}
%             \caption{图片标题\cite{viman}}
%         \end{figure}
%     \end{frame}

%     \begin{frame}
%         \frametitle{表与统计图}
%         \begin{multicols}{2}
%         \begin{table}
%             \caption{表格标题\cite{pgfplotstableman}}
%             \pgfplotstabletypeset[
%                 columns/Quick/.style={dec sep align},
%                 columns/Cocktail/.style={dec sep align},
%                 column type=r,
%                 % fixed zerofill,
%             ]{test.csv}
%         \end{table}
        
%         \begin{figure}
%             \input{testgraph.tex}
%             \caption{统计图标题\cite{pgfplotsman}}
%         \end{figure}
%         \end{multicols}
%     \end{frame}


% gbt=bibtex
\part{References 参考文献}
    \begin{frame}[allowframebreaks]
        \printbibliography[title=References 参考文献]    % gbt!=bibtex
        % \bibliography{ref.bib}             % gbt=bibtex
    \end{frame}

    \makebottom     % 创建尾页  % 非标准命令

\end{document}